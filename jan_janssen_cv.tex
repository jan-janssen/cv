%%%%%%%%%%%%%%%%%%%%%%%%%%%%%%%%%%%%%%%%%%%%%%%%%%%%%%%%%%%%%%%%%%%%%%%%%%%%%%%
% A clean template for an academic CV
%
% Uses tabularx to create two column entries (date and job/edu/citation).
% Defines commands to make adding entries simpler.
%
%%%%%%%%%%%%%%%%%%%%%%%%%%%%%%%%%%%%%%%%%%%%%%%%%%%%%%%%%%%%%%%%%%%%%%%%%%%%%%%

\documentclass[11pt, a4paper]{article}

% Full Unicode support for non-ASCII characters
\usepackage[utf8]{inputenc}

% Useful aliases
\newcommand{\TUKL}{Technical University of Kaiserslautern}
\newcommand{\MPIE}{Max-Planck-Institut f\"ur Eisenforschung}

% Identifying information
\newcommand{\Title}{Curriculum Vit\ae}
\newcommand{\FirstName}{Jan}
\newcommand{\LastName}{Janssen}
\newcommand{\Initials}{J}
\newcommand{\MyName}{\FirstName\ \LastName}
\newcommand{\Me}{\textbf{\Initials. \LastName}}  % For citations
\newcommand{\Email}{janssen@mpi-susmat.de}
\newcommand{\PersonalWebsite}{jan-janssen.com}
\newcommand{\LabWebsite}{mpie.de}
\newcommand{\Affiliation}{Max-Planck-Institute for Sustainable Materials}
\newcommand{\ORCID}{0000-0001-9948-7119}
\newcommand{\Address}{
  Max-Planck-Str. 1, 40237 D\"usseldorf, Germany
}

% Names for citing coauthors
\newcommand{\JN}{J. Neugebauer}
\newcommand{\RD}{R. Drautz}
\newcommand{\YL}{Y. Lysogorskiy}
\newcommand{\MT}{M. Todorova}

% Template configuration
%%%%%%%%%%%%%%%%%%%%%%%%%%%%%%%%%%%%%%%%%%%%%%%%%%%%%%%%%%%%%%%%%%%%%%%%%%%%%%%

% Disable hyphenation
\usepackage[none]{hyphenat}

% Control the font size
\usepackage{anyfontsize}

% Icon fonts (requires using xelatex or luatex)
\usepackage[fixed]{fontawesome5}
\usepackage{academicons}

% Template variables for styling
\newcommand{\TablePad}{\vspace{-0.4cm}}
\newcommand{\SoftwareTitle}[1]{{\bfseries #1}}
\newcommand{\TableTitle}[1]{{\fontsize{12pt}{0}\selectfont \itshape #1}}

% For fancy and multipage tables
\usepackage{tabularx}
\usepackage{ltablex}

% Define a new environment to place all CV entries in a 2-column table.
% Left column are the dates, right column the entries.
\usepackage{environ}
\NewEnviron{EntriesTable}{
\TablePad
\begin{tabularx}{\textwidth}{@{}p{0.12\textwidth}@{\hspace{0.02\textwidth}}p{0.86\textwidth}@{}}
  \BODY
\end{tabularx}
}

% Macros to add links and mark publications
\newcommand{\DOI}[1]{doi:\href{https://doi.org/#1}{#1}}
\newcommand{\DOILink}[1]{\href{https://doi.org/#1}{doi.org/#1}}
\newcommand{\Preprint}[1]{\newline • Preprint: \faFilePdf\ \DOILink{#1}}
\newcommand{\Youtube}[1]{\newline • Recording: \faYoutube\, \href{https://www.youtube.com/watch?v=#1}{youtube.com/watch?v=#1}}
\newcommand{\GitHub}[1]{\newline • Code: \faGithub\ \href{https://github.com/#1}{#1}}
\newcommand{\Role}[1]{\newline • Role: #1}
\newcommand{\Website}[1]{\newline • Website: \href{https://#1}{#1}}
\newcommand{\Slides}[1]{\newline • Slides: \faTv\ \href{https://#1}{#1}}
\newcommand{\SlidesDOI}[1]{\newline • Slides: \faTv\ \DOILink{#1}}
\newcommand{\PosterDOI}[1]{\newline • Poster: \faImage\ \DOILink{#1}}
\newcommand{\OA}{\aiOpenAccess\enspace}
\newcommand{\Invited}{\newline • \textbf{Invited talk}}

% Macros to set the year and duration on the left column
\newcommand{\Duration}[2]{\fontsize{10pt}{0}\selectfont #1 -- #2}
\newcommand{\Year}[1]{\fontsize{10pt}{0}\selectfont #1}
\newcommand{\Ongoing}{present}
%\newcommand{\Ongoing}{$\ast$}
\newcommand{\Future}{future}
\newcommand{\Review}{in review}
\newcommand{\Accepted}{accepted}
\newcommand{\Appointment}[5]{\textbf{#1} - #2\newline Research topic: #3\newline #4, #5}

% Define command to insert month name and year as date
\usepackage{datetime}
\newdateformat{monthyear}{\monthname[\THEMONTH], \THEYEAR}

% Set the page margins
\usepackage[a4paper,margin=1.5cm,includehead,headsep=5mm]{geometry}

% To get the total page numbers (\pageref{LastPage})
\usepackage{lastpage}

% No indentation
\setlength\parindent{0cm}

% Increase the line spacing
\renewcommand{\baselinestretch}{1.1}
% and the spacing between rows in tables
\renewcommand{\arraystretch}{1.5}

% Remove space between items in itemize and enumerate
\usepackage{enumitem}
\setlist{nosep}

% Use custom colors
\usepackage[usenames,dvipsnames]{xcolor}

% Set fonts. Requires compilation with xelatex
% \usepackage{fontspec}  % required to make older xelatex compile with UTF8

% Configure the font style for sections
\usepackage{sectsty}
\sectionfont{\vspace{0.5cm}\bfseries\fontsize{12pt}{0}\selectfont\uppercase}
\subsectionfont{\vspace{0.2cm}\mdseries\fontsize{12pt}{0}\selectfont\uppercase}

% Set the spacing for sections
%\usepackage{titlesec}
%\titlespacing{\section}{0pt}{0cm}{0.3cm}
%\titlespacing{\subsection}{0pt}{0.3cm}{0.3cm}

% Disable number of sections. Use this instead of "section*" so that the sections still
% appear as PDF bookmarks. Otherwise, would have to add the table of contents entries
% manually.
\makeatletter
\renewcommand{\@seccntformat}[1]{}
\makeatother

% Set fancy headers
\usepackage{fancyhdr}
\pagestyle{fancy}
\fancyhf{}
\chead{
  \fontsize{10pt}{12pt}\selectfont
  \MyName
  \hspace{0.2cm} -- \hspace{0.2cm}
  \Title
  \hspace{0.2cm} -- \hspace{0.2cm}
  \monthyear\today
}
\rhead{\fontsize{10pt}{0}\selectfont \thepage/\pageref*{LastPage}}
\renewcommand{\headrulewidth}{0pt}

% Metadata for the PDF output and control of hyperlinks
\usepackage[colorlinks=true]{hyperref}
\hypersetup{
  pdftitle={\MyName\ - \Title},
  pdfauthor={\MyName},
  linkcolor=blue,
  citecolor=blue,
  filecolor=black,
  urlcolor=MidnightBlue
}
%%%%%%%%%%%%%%%%%%%%%%%%%%%%%%%%%%%%%%%%%%%%%%%%%%%%%%%%%%%%%%%%%%%%%%%%%%%%%%%


\begin{document}

% No header for the first page
\thispagestyle{empty}

%%%%%%%%%%%%%%%%%%%%%%%%%%%%%%%%%%%%%%%%%%%%%%%%%%%%%%%%%%%%%%%%%%%%%%%%%%%%%%%
% HEADER
{\fontsize{22pt}{0}\selectfont\MyName}\\[-0.1cm]
\rule{\textwidth}{0.2pt}
\begin{minipage}[t]{0.595\textwidth}
  \Affiliation
  \\
  \Address
\end{minipage}
\begin{minipage}[t]{0.405\textwidth}
  \begin{flushright}
    Email: \href{mailto:\Email}{\Email}
    \\
    Website: \href{https://www.\PersonalWebsite}{\PersonalWebsite}
  \end{flushright}
\end{minipage}

%%%%%%%%%%%%%%%%%%%%%%%%%%%%%%%%%%%%%%%%%%%%%%%%%%%%%%%%%%%%%%%%%%%%%%%%%%%%%%%
\section{Professional Appointments}

\begin{EntriesTable}
  \Duration{2023}{\Ongoing}  &
  \Appointment{Group leader for Materials Informatics}{Computational Materials Design}{foundation models for sustainable materials}{Max-Planck-Institute for Sustainable Materials}{D\"usseldorf, Germany}
  \\
  \Year{2024}  &
  \Appointment{Invited Fellow}{Institute for Mathematical and Statistical Innovation (IMSI)}{data-driven materials informatics}{University of Chicago}{Chicago, IL, USA}
  \\
  \Duration{2022}{2023}  &
  \Appointment{Director's Postdoctoral Fellowship}{Theoretical Division (T-1)}{long-timescale simulations of materials with \textit{ab initio} accuracy}{Los Alamos National Laboratory}{Los Alamos, NM, USA}
  \\
  \Year{2023}   &
  \Appointment{Invited Postdoctoral Fellow}{Institute for Pure and Applied Mathematics (IPAM)}{new mathematics for the exascale: applications to materials science}{University of California}{Los Angeles, CA, USA}
  \\
  \Duration{2021}{2022}  &
  \Appointment{Postdoctoral Research Associate}{Theoretical Division (T-1)}{exascale atomistic capability for accuracy, length, and time }{Los Alamos National Laboratory}{Los Alamos, NM, USA}
  \\
  \Year{2017}   &
  \Appointment{Invited Fellow}{Institute for Pure and Applied Mathematics (IPAM)}{complex high-dimensional energy landscapes}{University of California}{Los Angeles, CA, USA}
  \\
  \Duration{2015}{2021}  &
  \Appointment{PhD Candidate}{Computational Materials Design}{computational phase studies and \textit{ab initio} thermodynamics}{\MPIE}{D\"usseldorf, Germany}
\end{EntriesTable}


%%%%%%%%%%%%%%%%%%%%%%%%%%%%%%%%%%%%%%%%%%%%%%%%%%%%%%%%%%%%%%%%%%%%%%%%%%%%%%%
\section{Education}

\begin{EntriesTable}
  \Duration{2015}{2021}  &
  \textbf{PhD in Theoretical Physics} - Paderborn University, Germany \newline
   • Thesis: pyiron - an integrated development environment for \textit{ab initio} thermodynamics \newline
   • Advisor: Prof. J\"org Neugebauer\newline
   • Grade: summa cum laude
  \\
  \Duration{2009}{2015}  &
  \textbf{Advanced Degree in Physics} - \TUKL, Germany \newline
   • Thesis: carbon in $\alpha$-iron-grainboundaries - an atomistic study of elastic properties\newline
   • Advisor: Prof. Herbert Urbassek\newline
   • Grade: thesis 1.0 (excellent) - total 1.6 (good)
\end{EntriesTable}
\pagebreak
%%%%%%%%%%%%%%%%%%%%%%%%%%%%%%%%%%%%%%%%%%%%%%%%%%%%%%%%%%%%%%%%%%%%%%%%%%%%%%%
\section{Funded Proposals}
\begin{EntriesTable}
%\Year{future}  &
\Year{2024} &
  \textbf{Principal Investigator} as part of the International Max Planck Research School for Sustainable Metallurgy - from Fundamentals to Engineering Materials
  \newline • Funding: 150k Euro 
  \\ &
  \textbf{Co-Principal Investigator} as part of the collaborative research center (CRC) 1394 ``Structural and chemical atomic complexity - from defect phase diagrams to material properties``
  \newline • Funding: 160k Euro 
  \\
\Year{2023} &
  \textbf{Principal Investigator} of ``workflows for machine learned interatomic potentials`` as part of an internal call for funding for Postdoctoral Fellows at Los Alamos National Laboratory
  \newline • Computing Hardware Funding: \$13k 
  \\ &
  \textbf{Co-Principal Investigator} of ``uncertainty propagation for multi-fidelity machine learned interatomic potentials`` as part of the summer school for applied machine learning at Los Alamos National Laboratory
  \newline • Funding: \$25k 
  \\
\Year{2022} &
  \textbf{Co-Principal Investigator} of ``development and deployment of a fully autonomous in \textit{silico} processing and materials discovery platform`` in collaboration with the department of Mechanical Engineering at Texas A\&M University
  \newline • Funding: \$176k 
  \\ &
  \textbf{Co-Principal Investigator} of ``transferability of interatomic machine learning potentials`` as part of the summer school for applied machine learning at Los Alamos National Laboratory
  \newline • Funding: \$25k 
  \\ &
  \textbf{Principal Investigator} of ``helping users build workflows with ontological typing`` project funded by a NumFOCUS small development grant to hire an independent contractor 
  \newline • Funding: \$6k 
\end{EntriesTable}

%%%%%%%%%%%%%%%%%%%%%%%%%%%%%%%%%%%%%%%%%%%%%%%%%%%%%%%%%%%%%%%%%%%%%%%%%%%%%%%
\section{Awards \& Honors}

\begin{EntriesTable}
  \Year{2024}  & 
  Winner of the ``2nd Annual Large Language Model Hackathon for Applications in Materials Science and Chemistry`` coupling a large language model to atomistic simulation
  \\ &
  Invited fellow at the Institute for Statistical and Mathematical Innovation (IMSI)
  \\
  \Year{2023}  &
  Invited postdoctoral fellow at the Institute for Pure and Applied Mathematics (IPAM)
  \\
  \Year{2022}  &
  Los Alamos National Laboratory (LANL) director's postdoctoral fellowship
  \\
  \Year{2021}  &
  PhD with highest distinction - summa cum laude
  \\
  \Year{2019}  &
  Runner-up for the Heinz Billing Award - a national biennial award for outstanding contributions to computational science by younger scientist without tenure 
  \\
  \Year{2017}  &
  Invited fellow at the Institute for Pure and Applied Mathematics (IPAM)
  \\
  \Year{2015}  &
  Scholarship of the \TUKL
\end{EntriesTable}
%%%%%%%%%%%%%%%%%%%%%%%%%%%%%%%%%%%%%%%%%%%%%%%%%%%%%%%%%%%%%%%%%%%%%%%%%%%%%%%
\pagebreak
\section{Publications}
Google Scholar, 03/25: Citations: 282

\begin{EntriesTable}
\Year{2025}  &
  \Me, M.G. Taylor, P. Yang, {\JN} and D. Perez. Executorlib – Up-scaling Python workflows for hierarchical heterogenous high-performance computing. \emph{Journal of Open Source Software}. Citations:~0
  \DOI{10.21105/joss.07782}.
  \GitHub{pyiron/executorlib}
  \\
  ~&
  Sk Md A.A. Alvi, \Me, D. Khatamsaz, D. Perez, D. Allaire and R. Arróyave. Hierarchical Gaussian process-based Bayesian optimization for materials discovery in high entropy alloy spaces. \emph{JACS Au}. Citations:~3
  \DOI{10.1016/j.actamat.2025.120908}.
  \GitHub{sheikhahnaf/DGP-BO}
  \\
\Year{2024}  &
  \Me, E. Makarov, T. Hickel, A.V. Shapeev and \JN. Automated optimization and uncertainty quantification of convergence parameters in plane wave density functional theory. \emph{npj Computational Materials}. Citations:~1
  \DOI{10.1038/s41524-024-01388-2}.
  \GitHub{pyiron/pyiron-dft-uncertainty}
  \\
  ~&
  T.J. Summers, M.G. Taylor, L.J. Augustine, \Me, D. Perez, E.R. Batista, and P. Yang. On the Importance of Configuration Search to the Predictivity of Lanthanide Selectivity. \emph{JACS Au}. Citations:~4
  \DOI{10.1021/jacsau.4c00770}.
  \\
  ~ &
  S. Menon, Y. Lysogorskiy, A. L. M. Knoll, N. Leimeroth, M. Poul, M. Qamar, \Me, M. Mrovec, J. Rohrer, K. Albe, J. Behler, R. Drautz, and \JN.
  From electrons to phase diagrams with classical and machine learning potentials automated workflows for materials science with pyiron.
  \emph{npj Computational Materials}. Citations:~4
  \DOI{10.1038/s41524-024-01441-0}.
  \GitHub{pyiron/potential\_publication}
  \\
  ~ &
  M. Tynes, M.G. Taylor, \Me, D. Burrill, D. Perez, P. Yang, and N. Lubbers.
  Linear graphlet models for accurate and interpretable cheminformatics.
  \emph{Digital Discovery}. Citations:~3
  \DOI{10.1039/D4DD00089G}.
  \GitHub{lanl/minervachem}
  \\
\Year{2023}  &
  M.G. Taylor, D.J. Burrill, \Me, E. Batista, D. Perez, and P. Yang.
  Architector: high-throughput cross-periodic table 3D complex builder. \emph{Nature Communications}.
  \DOI{10.1038/s41467-023-38169-2}. Citations:~30
  \GitHub{lanl/Architector}
  \\
  ~ &
  A. Rohskopf, C. Sievers, N. Lubbers, M. A. Cusentino, J. Goff, \Me, M. McCarthy, D. Montes de Oca Zapiain, S. Nikolov, K. Sargsyan, E. Sikorski, L. Williams, D. Sema, A. P. Thompson, and M. A. Wood.
  FitSNAP: Atomistic machine learning in LAMMPS. \emph{Journal of Open Source Software}.
  \DOI{10.21105/joss.05118}. Citations:~36
  \GitHub{FitSNAP/FitSNAP}
  \\
\Year{2021}  &
  L.F. Zhu, \Me, S. Ishibashi, F. Körmann, B. Grabowski and \JN.
  A fully automated approach to calculate the melting temperature of elemental crystals.
  \emph{Computational Materials Science}.
  \DOI{10.1016/j.commatsci.2020.110065}. Citations:~39
  \GitHub{pyiron/pyiron\_meltingpoint}
  \\
\Year{2020}  &
  T.D. Swinburne, \Me, \MT,  G. Simpson, P. Plechac, M. Luskin and \JN.
  Anharmonic free energy of lattice vibrations in fcc crystals from a mean field bond.
  \emph{Physical Review B}.
  \DOI{10.1103/PhysRevB.102.100101}. Citations:~8
  \GitHub{tomswinburne/BLaSA}
  \\
\Year{2019}  &
  \Me, S. Surendralal, \YL, \MT, T. Hickel, \RD{} and \JN.
  pyiron: an integrated development environment for computational materials science.
  \emph{Computational Materials Science}.
  \DOI{10.1016/j.commatsci.2018.07.043}. Citations:~117
  \GitHub{pyiron/pyiron}
  \\
  ~ &
  \YL, T. Hammerschmidt, \Me, \JN{} and \RD.
  Transferability of interatomic potentials for molybdenum and silicon.
  \emph{Modelling and Simulation in Materials Science and Engineering}.
  \DOI{10.1088/1361-651X/aafd13}. Citations:~24
  \\
\Year{2016}  &
  \Me, N. Gunkelmann and H. M. Urbassek.
  Influence of C concentration on elastic moduli of $\alpha^{\prime}-Fe_{1-x}C_{x}$ alloys.
  \emph{Philosophical Magazine}.
  \DOI{10.1080/14786435.2016.1170224}.
  Citations:~13
\end{EntriesTable} 

\subsection{Publications in review}
\begin{EntriesTable}
\Year{2025}  &
  M. Stricker, L. Banko, N. Sarazin, N. Siemer, \Me, L. Zhang, \JN and A. Ludwig. Computationally accelerated experimental materials characterization - drawing inspiration from high-throughput simulation workflows. 
  \DOI{10.48550/arXiv.2212.04804}.
  \\
\end{EntriesTable}

\subsection{Open-source Software}

\begin{EntriesTable}
  \Duration{2015}{\Ongoing} &
  \textbf{pyiron} - an integrated development environment for computational materials science
  \Role{Lead developer for a team of eight core developers}
  \GitHub{pyiron}
  \\
  \Duration{2018}{\Ongoing} &
  \textbf{Conda-forge} - community-led software distribution for the conda package manager
  \Role{Maintainer for materials science software}
  \newline • Contribution: over $900$ packages with a total of over $300$ million downloads
  \GitHub{conda-forge/staged-recipes}
  \\
  \Year{2024} &
  \textbf{LangSim} - Large language model for atomistic simulation
  \newline • Contribution: Lead the team of international researchers
  \GitHub{jan-janssen/LangSim}
  \\
  \Year{2022} &
  \textbf{Architector} - high-throughput cross-periodic table 3D complex builder
  \newline • Contribution: Parallelization of the chemical complex building using mpi4py
  \GitHub{lanl/Architector}
  \\
  \Year{2021} &
  \textbf{FitSNAP} - Software for generating SNAP machine-learning interatomic potentials
  \newline • Contribution: Implemented a python library interface for the Exascale computing project
  \GitHub{fitsnap/fitsnap}
\end{EntriesTable}
%%%%%%%%%%%%%%%%%%%%%%%%%%%%%%%%%%%%%%%%%%%%%%%%%%%%%%%%%%%%%%%%%%%%%%%%%%%%%%%
\newpage
\section{Workshops}

\begin{EntriesTable}
%\Year{future}  &
\Year{2021} &
  \textbf{Co-organizer} for the workshop on ``workflows for atomistic simulation``
  \newline
  \textit{\MPIE}, D\"usseldorf, Germany (online)
  \GitHub{pyiron-workshop/potentials-workshop-2021}
  \\
\Year{2020} &
  \textbf{Co-organizer} for the workshop on ``software tools from atomistics to phase diagrams``
  \newline
  \textit{Pennsylvania State University}, Pennsylvania, USA (online)
  \GitHub{pyiron-workshop/phasediagram-workshop-2020}
\end{EntriesTable}

%%%%%%%%%%%%%%%%%%%%%%%%%%%%%%%%%%%%%%%%%%%%%%%%%%%%%%%%%%%%%%%%%%%%%%%%%%%%%%%
\section{Supervision}

\begin{EntriesTable}
%\Year{future}  &
\Duration{2025}{\Ongoing} &
  \textbf{Prabhath Chilakalapudi} (PhD student): Machine learned interatomic potentials combining theoretical predictions and experimental references
  \Role{Supervisor - acquiring the funding and leading the supervision of the PhD student}
  \\
\Duration{2024}{\Ongoing} &
  \textbf{Haitham Wael Ali Awad Gaafer} (PhD student): Machine learned interatomic potentials for chemo-mechanics balancing computational efficiency and accuracy
  \Role{Supervisor - acquiring the funding and leading the supervision of the PhD student}
  \\
\Duration{2023}{\Ongoing} &
  \textbf{Sk Md Ahnaf Akif Alvi} (PhD student): Autonomous in silico processing and materials discovery for additive manufacturing
  \Role{Co-Supervisor - in collaboration with Prof. Arroyave at Texas A\&M University}
  \\
\Duration{2023}{2025} &
  \textbf{Ilgar Baghishov} (intern): Multi-fidelity machine learned interatomic potentials
  \Role{Primary Supervisor - acquiring the funding and leading the supervision of the intern}
  \\
\Duration{2022}{2023} &
  \textbf{Jason Blake Gibson} (intern): Transferability of interatomic machine learning potentials 
  \Role{Primary Supervisor - acquiring the funding and leading the supervision of the intern}
  \\
\Year{2018} &
  \textbf{Ankita Biswas} (student assistant): Calculation of vacancy formation energies with pyiron
  \Role{Primary Supervisor - for a student research project}
  \\
\Year{2017} &
  \textbf{Martin B\"ockmann} (student assistant): Monte Carlo sampling with pyiron
  \Role{Primary Supervisor - guiding the student as the first user of the pyiron software}
\end{EntriesTable}

%%%%%%%%%%%%%%%%%%%%%%%%%%%%%%%%%%%%%%%%%%%%%%%%%%%%%%%%%%%%%%%%%%%%%%%%%%%%%%%
\section{Presentations}

\subsection{Invited Talks}
\begin{EntriesTable}
\Year{2025}  &
  \Me.
  Predicting Sustainable Materials with Machine Learning,
  \emph{Nanomaterials: Computation, Theory, Machine Learning and Experiment},
  Telluride, Co, USA.
  \\
  ~ &
  {\Me} and \JN.
  Machine Learned Interatomic Potentials for Extreme Environments,
  \emph{Workshop on Models and Data for Plasma-Material Interactions in Fusion},
  Vienna, Austria.
  \\
  ~ &
  \Me.
  Predicting Sustainable Materials with Machine Learned Interatomic Potentials,
  \emph{AAPALI-Psi-K},
  Pune, India.
  \\
  ~ &
  \Me.
  pyiron – Workflows for data-driven Materials Science,
  \emph{NHR4CES Community Workshop},
  Darmstadt, Germany (online).
  \\
  ~ &
  \Me.
  Predicting Sustainable Materials with Machine Learning,
  \emph{Machine learning with computational materials science data for modelling nanocrystal catalysts},
  Freiburg, Germany.
  \\
\Year{2024}  &
  \Me.
  How to use machine learning in the discovery and design of materials for the future?,
  \emph{Michigan Institute for Computational Discovery \& Engineering},
  Michigan, MI, USA.
  \\
  ~ &
  \Me.
  Foundation Models for Materials Discovery,
  \emph{Summer School at Center for Scientific Foundation Models},
  Michigan, MI, USA (online).
  \\
  ~ &
  \Me.
  pyiron: Simulation Workflows for Data-Driven Materials Design,
  \emph{Center for Nonlinear Studies (CNLS) Seminar Los Alamos National Laboratory},
  Los Alamos, NM, USA.
  \\
  ~ &
  \Me.
  pyiron: workflows for the development and assessment of interatomic potentials,
  \emph{Workshop on Machine Learning Force Fields at the institute for mathematical and statistical innovation (IMSI)},
  Chicago, Il, USA.
  \\
\Year{2023}  &
  \Me and \JN.
  Rapid prototyping and up-scaling atomistic workflows with pyiron ,
  \emph{Ab initio Description of Iron and Steel (ADIS): Digitalization and Workflows},
  Kreuth, Germany.
  \\
  ~ &
  \Me.
  Packaging Scientific Software with Conda-Forge,
  \emph{Workshop at the Institute for Pure and Applied Math (IPAM)},
  Los Angeles, CA, USA.
  \\
  ~ &
  {\Me} and D. Perez.
  Up-scaling atomistic simulation workflows with pyiron,
  \emph{Workshop at the Institute for Pure and Applied Math (IPAM)},
  Los Angeles, CA, USA.
  \\
  ~ &
  \Me, M. G. Taylor, P. Yang and D. Perez.
  Tutorial on high-throughput screening for chemical science with pyiron,
  \emph{2nd International Workshop on Theory Frontiers in Actinide Science : Chemistry \& Materials},
  Santa Fe, NM, USA.
  \\
\Year{2022}  &
  \Me, T. Hickel and \JN.
  Automated atomistic calculation of thermodynamic and thermophysical data,
  \emph{DPG Fall Meeting},
  Regensburg, Germany.
  \\
  ~ &
  \Me.
  pyiron - an integrated development environment (IDE) for scientific workflows at scale,
  \emph{Department of Energy Python Exchange},
  New York, NY, USA (online).
  \\
  ~ &
  \Me.
  Up-scaling simulation protocols with pyiron,
  \emph{Platform Material Digital Workflow Meeting 2022},
  Berlin, Germany (online).
  \\
  ~ &
  \Me.
  pyiron - an integrated development environment (IDE) for materials science,
  \emph{Special Interest Group Data Infrastructure (SIGDIUS) Seminar},
  Stuttgart, Germany (online).
  \\
\Year{2021}  &
  {\Me} and \JN.
  pyiron - an integrated development environment for materials science,
  \emph{CECAM Workshop: Simulation Workflows in Materials Modelling (SWiMM)},
  Lausanne, Switzerland (online).
  \\
\Year{2020}  &
  \Me, T. Hickel and \JN.
  Uncertainty quantification for \textit{ab initio} thermodynamics,
  \emph{Group Seminar Professor \RD},
  Bochum, Germany (online).
  \\
  ~ &
  \Me, T. Hickel and \JN.
  Automated \textit{ab initio} determination of materials properties at finite temperatures with pyiron,
  \emph{NIST Workshop: Atomistic simulations for industrial needs},
  Gaitgersburg, MD, USA (online).
  \\
\Year{2019}  &
  \Me, T. Hickel and \JN.
  Automated \textit{ab initio} determination of materials properties at finite temperatures with pyiron,
  \emph{Center for Nonlinear Studies (CNLS) Seminar at Los Alamos National Laboratory},
  Los Alamos, NM, USA.
  \\
  ~ &
  \Me, T. Hickel and \JN.
  pyiron - an integrated development environment for computational materials science,
  \emph{Group Seminar Professor G. Kresse},
  Vienna, Austria.
  \\
\end{EntriesTable}
\subsection{Talks at International Conferences}
\begin{EntriesTable}
\Year{2025}  &
  {\Me} and \JN.
  Inverse Materials Design with Large Language Models,
  \emph{DPG Spring Meeting},
  Regensburg, Germany.
  \\
\Year{2024}  &
  \Me, \JN and D. Perez.
  Potential in a day - up-scale the fitting of machine learning potentials for the Exascale,
  \emph{IPAM Workshop},
  Los Angeles, CA, USA.
  \\
  ~ &
  {\Me} and \JN.
   pyiron - Simulation Workflows for the Design of Sustainable Materials,
  \emph{MRS Fall Meeting},
  Boston, MA, USA.
  \\
  ~ &
  M. Stricker, L. Banko, N. Sarazin, \Me, N. Siemer, \JN and A. Ludwig.
  Computationally Accelerated Experimental Materials Characterization Using pyiron,
  \emph{Multiscale Materials Modelling (MMM) conference},
  Prague, CZ.
  \\
  ~ &
  {\Me} and \JN.
  Automated Workflows to Construct and Validate Machine-Learned Interatomic Potentials,
  \emph{Multiscale Materials Modelling (MMM) conference},
  Prague, CZ.
  \\
  ~ &
  I. Baghishov, \Me, G. Henkelman and D. Perez.
  Investigating the uncertainty in multi-fidelity machine learning interatomic potentials,
  \emph{TMS Spring Meeting},
  Orlando, Fl, USA.
  \\
  ~ &
  {\Me} and D. Perez.
  Transferable Machine Learning Potentials for Extreme Environments,
  \emph{TMS Spring Meeting},
  Orlando, Fl, USA.
  \\
\Year{2023}  &
  {\Me} and D. Perez.
  Parameter studies for interatomic potentials using LAMMPS and pyiron,
  \emph{LAMMPS Virtual Workshop and Symposium 2023},
  Philadelphia, PA, USA (online).
  \\
  ~ &
  {\Me} and D. Perez.
  Enabling long timescale molecular dynamics simulation with ab initio precision,
  \emph{MRS Spring Meeting},
  San Francisco, CA, USA.
  \\
  ~ &
  {\Me} and D. Perez.
  pyiron an integrated development environment for the development and assessment of interatomic models,
  \emph{11th Annual Mach Conference},
  Baltimore, MD, USA.
  \\
  ~ &
  {\Me} and D. Perez.
  Enabling long timescale molecular dynamics simulation with ab initio precision,
  \emph{TMS Spring Meeting},
  San Diego, CA, USA.
  \\
\Year{2022}  &
  \Me, T. Hickel and \JN.
  Predicting melting temperatures from bulk properties with pyiron,
  \emph{Multiscale Materials Modelling (MMM) conference},
  Baltimore, MD, USA.
  \\
  ~ &
  \Me, M. G. Taylor, P. Yang and D. Perez.
  Screening of ligand-metal-complexes for separation science with pyiron,
  \emph{ACS Fall Meeting},
  Chicago, IL, USA.
  \\
  ~ &
  \Me. 
  pyiron – an integrated development environment (IDE) for scientific workflows,
  \emph{Scientific Computing with Python (SciPy)},
  Austin, TX, USA.
  \\
  ~ &
  \Me.
  Predicting melting temperatures from bulk properties with pyiron,
  \emph{Artificial Intelligence for Materials Science (AIMS)},
  Gaithersburg, MD, USA (online).
  \\
\Year{2019}  &
  \Me, T. Hickel and \JN.
  Automated uncertainty quantification for \textit{ab initio} thermodynamics,
  \emph{MRS Fall Meeting},
  Boston, MA, USA.
  \\
  ~ &
  \Me, T. Hickel and \JN.
  Automated sensitivity analysis for high-throughput \textit{ab initio} calculations,
  \emph{IPAM Workshop},
  Los Angeles, CA, USA.
  \\
  ~ &
  \Me, T. Hickel and \JN.
  Automated error analysis and control for \textit{ab initio} calculations,
  \emph{DPG Spring Meeting},
  Regensburg, Germany.
  \\
  ~ &
  \Me, T. Hickel and \JN.
  Automated sensitivity analysis for high-throughput \textit{ab initio} calculations,
  \emph{TMS Spring Meeting},
  San Antonio, TX, USA.
  \\
\Year{2018}  &
  \Me, T. Hickel and \JN.
  Generation of \textit{ab initio} datasets with predefined precision using uncertainty quantification,
  \emph{DPG Spring Meeting},
  Berlin, Germany.
  \\
\Year{2017}  &
  \Me, T. Hickel and \JN.
  Towards an uncertainty quantification for \textit{ab initio} thermodynamics,
  \emph{MRS Fall Meeting},
  Boston, MA, USA.
  \\
  ~&
  \Me, T. Hickel and \JN.
  Sensitivity analyis for large sets of density functional theory calculations,
  \emph{DPG Spring Meeting},
  Dresden, Germany.
  \\
  ~&
  \Me, T. Hickel and \JN.
  Automated convergence and error analyses for high-precision DFT calculations,
  \emph{TMS Spring Meeting},
  San Diego, CA, USA.
  \\
\Year{2016}  &
  \Me, T. Hickel and \JN.
  Automated convergence checks with the python based library pyiron,
  \emph{DPG Spring Meeting},
  Regensburg, Germany.
  \\
  ~&
  \Me, T. Hickel and \JN.
  Automated convergence checks with the python based workbench pyiron,
  \emph{TMS Spring Meeting},
  Nashville, TN, USA.
  \\
\end{EntriesTable}

\end{document}
