%%%%%%%%%%%%%%%%%%%%%%%%%%%%%%%%%%%%%%%%%%%%%%%%%%%%%%%%%%%%%%%%%%%%%%%%%%%%%%%
% A clean template for an academic CV
%
% Uses tabularx to create two column entries (date and job/edu/citation).
% Defines commands to make adding entries simpler.
%
%%%%%%%%%%%%%%%%%%%%%%%%%%%%%%%%%%%%%%%%%%%%%%%%%%%%%%%%%%%%%%%%%%%%%%%%%%%%%%%

\documentclass[11pt, a4paper]{article}

% Full Unicode support for non-ASCII characters
\usepackage[utf8]{inputenc}

% Useful aliases
\newcommand{\TUKL}{Technical University of Kaiserslautern}
\newcommand{\MPIE}{Max-Planck-Institut f\"ur Eisenforschung}

% Identifying information
\newcommand{\Title}{Curriculum Vit\ae}
\newcommand{\FirstName}{Jan}
\newcommand{\LastName}{Janssen}
\newcommand{\Initials}{J}
\newcommand{\MyName}{\FirstName\ \LastName}
\newcommand{\Me}{\textbf{\Initials. \LastName}}  % For citations
\newcommand{\Email}{jan.janssen@outlook.com}
\newcommand{\PersonalWebsite}{jan-janssen.com}
\newcommand{\LabWebsite}{lanl.gov}
\newcommand{\Affiliation}{Theoretical Division \\ Los Alamos National Laboratory}
\newcommand{\ORCID}{0000-0001-9948-7119}
\newcommand{\Address}{
  Bikini Atoll Rd., SM 30 \\ Los Alamos, NM 87545, USA
}

% Names for citing coauthors
\newcommand{\JN}{J. Neugebauer}
\newcommand{\RD}{R. Drautz}
\newcommand{\YL}{Y. Lysogorskiy}
\newcommand{\MT}{M. Todorova}

% Template configuration
%%%%%%%%%%%%%%%%%%%%%%%%%%%%%%%%%%%%%%%%%%%%%%%%%%%%%%%%%%%%%%%%%%%%%%%%%%%%%%%

% Disable hyphenation
\usepackage[none]{hyphenat}

% Control the font size
\usepackage{anyfontsize}

% Icon fonts (requires using xelatex or luatex)
\usepackage[fixed]{fontawesome5}
\usepackage{academicons}

% Template variables for styling
\newcommand{\TablePad}{\vspace{-0.4cm}}
\newcommand{\SoftwareTitle}[1]{{\bfseries #1}}
\newcommand{\TableTitle}[1]{{\fontsize{12pt}{0}\selectfont \itshape #1}}

% For fancy and multipage tables
\usepackage{tabularx}
\usepackage{ltablex}

% Define a new environment to place all CV entries in a 2-column table.
% Left column are the dates, right column the entries.
\usepackage{environ}
\NewEnviron{EntriesTable}{
\TablePad
\begin{tabularx}{\textwidth}{@{}p{0.12\textwidth}@{\hspace{0.02\textwidth}}p{0.86\textwidth}@{}}
  \BODY
\end{tabularx}
}

% Macros to add links and mark publications
\newcommand{\DOI}[1]{doi:\href{https://doi.org/#1}{#1}}
\newcommand{\DOILink}[1]{\href{https://doi.org/#1}{doi.org/#1}}
\newcommand{\Preprint}[1]{\newline • Preprint: \faFilePdf\ \DOILink{#1}}
\newcommand{\Youtube}[1]{\newline • Recording: \faYoutube\, \href{https://www.youtube.com/watch?v=#1}{youtube.com/watch?v=#1}}
\newcommand{\GitHub}[1]{\newline • Code: \faGithub\ \href{https://github.com/#1}{#1}}
\newcommand{\Role}[1]{\newline • Role: #1}
\newcommand{\Website}[1]{\newline • Website: \href{https://#1}{#1}}
\newcommand{\Slides}[1]{\newline • Slides: \faTv\ \href{https://#1}{#1}}
\newcommand{\SlidesDOI}[1]{\newline • Slides: \faTv\ \DOILink{#1}}
\newcommand{\PosterDOI}[1]{\newline • Poster: \faImage\ \DOILink{#1}}
\newcommand{\OA}{\aiOpenAccess\enspace}
\newcommand{\Invited}{\newline • \textbf{Invited talk}}

% Macros to set the year and duration on the left column
\newcommand{\Duration}[2]{\fontsize{10pt}{0}\selectfont #1 -- #2}
\newcommand{\Year}[1]{\fontsize{10pt}{0}\selectfont #1}
\newcommand{\Ongoing}{present}
%\newcommand{\Ongoing}{$\ast$}
\newcommand{\Future}{future}
\newcommand{\Review}{in review}
\newcommand{\Accepted}{accepted}
\newcommand{\Appointment}[4]{\textbf{#1} \newline #2 \newline #3 \newline #4}

% Define command to insert month name and year as date
\usepackage{datetime}
\newdateformat{monthyear}{\monthname[\THEMONTH], \THEYEAR}

% Set the page margins
\usepackage[a4paper,margin=1.5cm,includehead,headsep=5mm]{geometry}

% To get the total page numbers (\pageref{LastPage})
\usepackage{lastpage}

% No indentation
\setlength\parindent{0cm}

% Increase the line spacing
\renewcommand{\baselinestretch}{1.1}
% and the spacing between rows in tables
\renewcommand{\arraystretch}{1.5}

% Remove space between items in itemize and enumerate
\usepackage{enumitem}
\setlist{nosep}

% Use custom colors
\usepackage[usenames,dvipsnames]{xcolor}

% Set fonts. Requires compilation with xelatex
\usepackage{fontspec}  % required to make older xelatex compile with UTF8

% Configure the font style for sections
\usepackage{sectsty}
\sectionfont{\vspace{0.5cm}\bfseries\fontsize{12pt}{0}\selectfont\uppercase}
\subsectionfont{\vspace{0.2cm}\mdseries\fontsize{12pt}{0}\selectfont\uppercase}

% Set the spacing for sections
%\usepackage{titlesec}
%\titlespacing{\section}{0pt}{0cm}{0.3cm}
%\titlespacing{\subsection}{0pt}{0.3cm}{0.3cm}

% Disable number of sections. Use this instead of "section*" so that the sections still
% appear as PDF bookmarks. Otherwise, would have to add the table of contents entries
% manually.
\makeatletter
\renewcommand{\@seccntformat}[1]{}
\makeatother

% Set fancy headers
\usepackage{fancyhdr}
\pagestyle{fancy}
\fancyhf{}
\chead{
  \fontsize{10pt}{12pt}\selectfont
  \MyName
  \hspace{0.2cm} -- \hspace{0.2cm}
  \Title
  \hspace{0.2cm} -- \hspace{0.2cm}
  \monthyear\today
}
\rhead{\fontsize{10pt}{0}\selectfont \thepage/\pageref*{LastPage}}
\renewcommand{\headrulewidth}{0pt}

% Metadata for the PDF output and control of hyperlinks
\usepackage[colorlinks=true]{hyperref}
\hypersetup{
  pdftitle={\MyName\ - \Title},
  pdfauthor={\MyName},
  linkcolor=blue,
  citecolor=blue,
  filecolor=black,
  urlcolor=MidnightBlue
}
%%%%%%%%%%%%%%%%%%%%%%%%%%%%%%%%%%%%%%%%%%%%%%%%%%%%%%%%%%%%%%%%%%%%%%%%%%%%%%%


\begin{document}

% No header for the first page
\thispagestyle{empty}

%%%%%%%%%%%%%%%%%%%%%%%%%%%%%%%%%%%%%%%%%%%%%%%%%%%%%%%%%%%%%%%%%%%%%%%%%%%%%%%
% HEADER
{\fontsize{22pt}{0}\selectfont\MyName}\\[-0.1cm]
\rule{\textwidth}{0.2pt}
\begin{minipage}[t]{0.595\textwidth}
  \Affiliation
  \\
  \Address
\end{minipage}
\begin{minipage}[t]{0.405\textwidth}
  \begin{flushright}
  Last updated: \monthyear\today
  \\
    ORCID: \href{https://orcid.org/\ORCID}{\ORCID}
    \\
    Email: \href{mailto:\Email}{\Email}
    \\
    Research lab: \href{https://www.\LabWebsite}{\LabWebsite}
    \\
    Website: \href{https://www.\PersonalWebsite}{\PersonalWebsite}
  \end{flushright}
\end{minipage}

%%%%%%%%%%%%%%%%%%%%%%%%%%%%%%%%%%%%%%%%%%%%%%%%%%%%%%%%%%%%%%%%%%%%%%%%%%%%%%%
\section{Professional Appointments}

\begin{EntriesTable}
  \Duration{2021}{\Ongoing}  &
  \Appointment{Postdoctoral Research Associate}{Theoretical Division (T-1)}{Los Alamos National Laboratory}{Los Alamos, NM, USA}
  \\
  \Duration{2015}{2021}  &
  \Appointment{PhD Candidate}{Computational Materials Design}{\MPIE}{D\"usseldorf, Germany}
  \\
  \Year{2017}   &
  \Appointment{Invited Fellow}{Institute for Pure and Applied Mathematics}{University of California}{Los Angeles, CA, USA}
\end{EntriesTable}


%%%%%%%%%%%%%%%%%%%%%%%%%%%%%%%%%%%%%%%%%%%%%%%%%%%%%%%%%%%%%%%%%%%%%%%%%%%%%%%
\section{Education}

\begin{EntriesTable}
  \Duration{2015}{2021}  &
  \textbf{PhD in Theoretical Physics}, Paderborn University, Germany
  \\
  \Duration{2009}{2015}  &
  \textbf{Advanced degree in Theoretical Physics (Master's equivalent)}, \TUKL, Germany
\end{EntriesTable}


%%%%%%%%%%%%%%%%%%%%%%%%%%%%%%%%%%%%%%%%%%%%%%%%%%%%%%%%%%%%%%%%%%%%%%%%%%%%%%%
\section{Awards \& Honors}

\begin{EntriesTable}
  \Year{2019}  &
  Runner-up for the Heinz Billing Award of 2019
\end{EntriesTable}


%%%%%%%%%%%%%%%%%%%%%%%%%%%%%%%%%%%%%%%%%%%%%%%%%%%%%%%%%%%%%%%%%%%%%%%%%%%%%%%
\section{Publications}

\begin{EntriesTable}
\Year{2021}  &
  L.F. Zhu, \Me, S. Ishibashi, F. Körmann, B. Grabowski and \JN.
  A fully automated approach to calculate the melting temperature of elemental crystals.
  \emph{Computational Materials Science}.
  \DOI{10.1016/j.commatsci.2020.110065}.
  \GitHub{pyiron/pyiron\_meltingpoint}
  \\
\Year{2020}  &
  T.D. Swinburne, \Me, \MT,  G. Simpson, P. Plechac, M. Luskin and \JN.
  Anharmonic free energy of lattice vibrations in fcc crystals from a mean field bond.
  \emph{Physical Review B}.
  \DOI{10.1103/PhysRevB.102.100101}.
  \GitHub{tomswinburne/BLaSA}
  \\
\Year{2019}  &
  \Me, S. Surendralal, \YL, \MT, T. Hickel, \RD{} and \JN.
  pyiron: an integrated development environment for computational materials science.
  \emph{Computational Materials Science}.
  \DOI{10.1016/j.commatsci.2018.07.043}.
  \GitHub{pyiron}
  \\
  ~ &
  \YL, T. Hammerschmidt, \Me, \JN{} and \RD.
  Transferability of interatomic potentials for molybdenum and silicon.
  \emph{Modelling and Simulation in Materials Science and Engineering}.
  \DOI{10.1088/1361-651X/aafd13}
  \\
\Year{2016}  &
  \Me, N. Gunkelmann and H. M. Urbassek.
  Influence of C concentration on elastic moduli of $\alpha^{\prime}-Fe_{1-x}C_{x}$ alloys.
  \emph{Philosophical Magazine}.
  \DOI{10.1080/14786435.2016.1170224}
\end{EntriesTable}


\subsection{Open-source Software}

\begin{EntriesTable}
  \Duration{2015}{\Ongoing} &
  \textbf{pyiron}
  \newline
  An integrated development environment for computational materials science
  \Role{Project Lead Developer}
  \GitHub{pyiron}
  \Website{pyiron.org}
  \\
  \Duration{2018}{\Ongoing} &
  \textbf{Conda-Forge}
  \newline
  A community-led collection of recipes, build infrastructure and distributions for the conda package manager.
  \Role{Maintainer for 400+ Materials Science Software Packages}
  \GitHub{conda-forge}
  \Website{conda-forge.org}
\end{EntriesTable}


%%%%%%%%%%%%%%%%%%%%%%%%%%%%%%%%%%%%%%%%%%%%%%%%%%%%%%%%%%%%%%%%%%%%%%%%%%%%%%%
\section{Workshops}

\begin{EntriesTable}
%\Year{future}  &
\Year{2021} &
  Workflows for Atomistic Simulation
  \newline
  \textit{\MPIE}, D\"usseldorf, Germany (online).
  \GitHub{pyiron/potentials-workshop-2021}
  \\
\Year{2020} &
  Software Tools from Atomistics to Phase Diagrams
  \newline
  \textit{\MPIE}, D\"usseldorf, Germany (online).
  \GitHub{pyiron/phasediagram-workshop-2020}
\end{EntriesTable}


%%%%%%%%%%%%%%%%%%%%%%%%%%%%%%%%%%%%%%%%%%%%%%%%%%%%%%%%%%%%%%%%%%%%%%%%%%%%%%%
\section{Presentations}

\subsection{Invited Talks}
\begin{EntriesTable}
\Year{2022}  &
  \Me.
  Up-scaling simulation protocols with pyiron,
  \emph{Platform Material Digital Workflow Meeting 2022},
  Berlin, Germany (online).
  \\
  ~ &
  \Me.
  pyiron - an integrated development environment (IDE) for materials science,
  \emph{Special Interest Group Data Infrastructure (SIGDIUS) Seminar},
  Stuttgart, Germany (online).
  \\
\Year{2021}  &
  \Me, T. Hickel, \JN.
  pyiron - an integrated development environment for materials science,
  \emph{CECAM Workshop: Simulation Workflows in Materials Modelling (SWiMM)},
  Lausanne, Switzerland (online).
  \\
\Year{2020}  &
  \Me, T. Hickel, \JN.
  Uncertainty quantification for ab initio thermodynamics,
  \emph{Group Seminar Professor \RD},
  Bochum, Germany (online).
  \\
  ~ &
  \Me, T. Hickel, \JN.
  Automated ab-initio determination of materials properties at finite temperatures with pyiron,
  \emph{NIST Workshop: Atomistic simulations for industrial needs},
  Gaitgersburg, MD, USA (online).
  \\
\Year{2019}  &
  \Me, T. Hickel, \JN.
  Automated ab-initio determination of materials properties at finite temperatures with pyiron,
  \emph{CLNS Seminar Los Alamos National Laboratory},
  Los Alamos, NM, USA.
  \\
  ~ &
  \Me, T. Hickel, \JN.
  pyiron - an integrated development environment for computational materials science,
  \emph{Group Seminar Professor G. Kresse},
  Vienna, Austria.
  \\
\end{EntriesTable}

\subsection{Talks at International Conferences}
\begin{EntriesTable}
\Year{2019}  &
  \Me, T. Hickel, \JN.
  Automated uncertainty quantification for ab initio thermodynamics,
  \emph{MRS Fall Meeting},
  Boston, MA, USA.
  \\
  ~ &
  \Me, T. Hickel, \JN.
  Automated sensitivity analysis for high-throughput ab initio calculations,
  \emph{IPAM Workshop},
  Los Angeles, CA, USA.
  \\
  ~ &
  \Me, T. Hickel, \JN.
  Automated error analysis and control for ab initio calculations,
  \emph{DPG Spring Meeting},
  Regensburg, Germany.
  \\
  ~ &
  \Me, T. Hickel, \JN.
  Automated sensitivity analysis for high-throughput ab initio calculations,
  \emph{TMS Spring Meeting},
  San Antonio, TX, USA.
  \\
\Year{2018}  &
  \Me, T. Hickel, \JN.
  Generation of ab initio datasets with predefined precision using uncertainty quantification,
  \emph{DPG Spring Meeting},
  Berlin, Germany.
  \\
\Year{2017}  &
  \Me, T. Hickel, \JN.
  Towards an uncertainty quantification for ab initio thermodynamics,
  \emph{MRS Fall Meeting},
  Boston, MA, USA.
  \\
  ~&
  \Me, T. Hickel, \JN.
  Sensitivity analyis for large sets of density functional theory calculations,
  \emph{DPG Spring Meeting},
  Dresden, Germany.
  \\
  ~&
  \Me, T. Hickel, \JN.
  Automated convergence and error analyses for high-precision DFT calculations,
  \emph{TMS Spring Meeting},
  San Diego, CA, USA.
  \\
\Year{2016}  &
  \Me, T. Hickel, \JN.
  Automated convergence checks with the python based library pyiron,
  \emph{DPG Spring Meeting},
  Regensburg, Germany.
  \\
  ~&
  \Me, T. Hickel, \JN.
  Automated convergence checks with the python based workbench pyiron,
  \emph{TMS Spring Meeting},
  Nashville, TN, USA.
  \\
\end{EntriesTable}

\end{document}
